\newpage
\section*{Exploration Questions}
\addcontentsline{toc}{section}{Exploration Questions}

\section{Span}

Let $\vec u=\mat{1\\1}$ and $\vec v=\mat{-1\\2}$. Can the vectors $\vec w=\mat{2\\5}$ be obtained
as a linear combination of $\vec u$ and $\vec v$?

By drawing a picture, the answer appears to be \emph{yes}.

XXX Figure

Algebraically, we can use the definition of \emph{linear combination} to set up a system of equations.
We know $\vec w$ can be expressed as a linear combination of $\vec u$ and $\vec v$ if and only if 
the vector equation
\[
	\vec w = \mat{2\\5}=\alpha\mat{1\\1}+\beta\mat{-1\\2}=\alpha \vec u+\beta \vec v
\]
has a solution. By inspection, we see $\alpha=3$ and $\beta=1$ solve this equation.

After initial success, we might be tempted to ask the following:
\emph{starting at the origin, what are all the locations in $\R^2$ that can be obtained
as a linear combination of $\vec u$ and $\vec v$?} Geometrically, it appears
any location can be reached. To verify this algebraically, we consider the vector equation
\begin{equation}
	\label{EQSPAN1}
	\vec x=\mat{x\\y} = \alpha\mat{1\\1}+\beta\mat{-1\\2} = \alpha\vec u+\beta\vec v.
\end{equation}
Here $\vec x$ represents an arbitrary point in $\R^2$. Thus, if equation \eqref{EQSPAN1} always
has a solution, any vector in $\R^2$ can be obtained as a linear combination of $\alpha$ and $\beta$.

We can solve this equation for $\alpha$ and $\beta$ by considering the equations arising from the
first and second components. Namely,
\begin{alignat*}{3}
	x &{}={}& \alpha &{}+{}& \beta\\
	y &{}={}& \alpha &{}-{}& 2\beta.
\end{alignat*}
Subtracting the second equation from the first, we get $x-y=3\beta$ and so $\beta=(x-y)/3$. Plugging 
$\beta$ into the first equation and solving, we get $\alpha=(2x+y)/3$. Thus, equation \eqref{EQSPAN1}
\emph{always} has a solution.

There is a formal term for the set of vectors that can be obtained as linear combinations: \emph{span}.

\begin{definition}[Span]
	Let $\mathcal X$ be a set of vectors. The \emph{span} of $\mathcal X$, written $\Span \mathcal X$,
	is the set of all linear combinations of vectors in $\mathcal X$. Formally,
	\[
	\Span \mathcal X = \Set{\vec x\given \vec x = \alpha_1\vec v_1+\alpha_2\vec v_2+
	\cdots+\alpha_n\vec v_n\text{ for some }\vec v_1,\ldots,\vec v_n\in\mathcal X
	\text{ and scalars }\alpha_1,\ldots,\alpha_n}
	\]
\end{definition}

We just showed above that $\Span\Set*{\mat{1\\1},\mat{-1\\2}}=\R^2$.

\begin{example}
	Let $\vec u=\mat{-1\\2}$ and $\vec v=\mat{1\\-2}$. Find $\Span\Set{\vec u,\vec v}$.

	XXX Finish
\end{example}

The objects that arise from spans are familiar. If $\vec v\neq\vec 0$, then $\Span\Set{\vec v}$
is the line with direction vector $\vec v$ through the origin. If $\vec v,\vec w\neq \vec 0$ and
aren't parallel, $\Span\Set{\vec v,\vec w}$ is a plane through the origin. In fact, vector form of
a line or a plane is nothing more than a \emph{translated span}.

\section{Linear Independence}

